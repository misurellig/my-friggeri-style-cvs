% !TEX encoding = UTF-8 Unicode
%%%%%%%%%%%%%%%%%%%%%%%%%%%%%%%%%%%%%%%%%
% Friggeri Resume/CV
% XeLaTeX Template
% Version 1.0 (5/5/13)
%
% This template has been downloaded from:
% http://www.LaTeXTemplates.com
%
% Original author:
% Adrien Friggeri (adrien@friggeri.net)
% https://github.com/afriggeri/CV
%
% License:
% CC BY-NC-SA 3.0 (http://creativecommons.org/licenses/by-nc-sa/3.0/)
%
% Important notes:
% This template needs to be compiled with XeLaTeX and the bibliography, if used,
% needs to be compiled with biber rather than bibtex.
%
%%%%%%%%%%%%%%%%%%%%%%%%%%%%%%%%%%%%%%%%%

%\documentclass[]{friggeri-cv} % Add 'print' as an option into the square bracket to remove colors from this template for printing
\documentclass[]{friggeri-cv} % Add 'print' as an option into the square bracket to remove colors from this template for printing

%\addbibresource{bibliography.bib} % Specify the bibliography file to include publications

\begin{document}

\header{Giuseppe}{ Misurelli}{DevOps Engineer} % Your name and current job title/field

%----------------------------------------------------------------------------------------
%	SIDEBAR SECTION
%----------------------------------------------------------------------------------------

\begin{aside} % In the aside, each new line forces a line break
\section{Contact}
1 Via Battibecco
40123 Bologna
Italy
~
+39 33 83 49 25 19
~
\href{mailto:giuseppe.misurelli@gmail.com}{gmail: giuseppe.misurelli}
\href{https://it.linkedin.com/pub/giuseppe-misurelli/2/820/537}{linkedin: Giuseppe Misurelli}
\section{Languages}
Italian: native
English: full professional proficiency
\section{Technologies}
Linux, Windows
KVM, oVirt
Amazon AWS, OpenStack
Ansible, Chef
Terraform, Packer
Jenkins
Nagios, Sensu, Zabbix
Elasticsearch, Logstash, Kibana
Apache, HaProxy
PostgreSQL, MySQL
Python, Ruby
Vagrant, Git
\end{aside}

%----------------------------------------------------------------------------------------
%   SUMMARY SECTION
%----------------------------------------------------------------------------------------
% \section{Summary}
% \textbf{} • 10+ years full-time working experience in IT operation\\
% \textbf{} • Coordinating the local delivery team providing deployment, monitoring and log analysis as a service for product teams\\
% \textbf{} • Design processes for node and service deployment and orchestration management\\
% \textbf{} • Ability to deal with security information and incident response\\
% \textbf{} • Re-use and adapt standards for the definition of policy and best practices\\
% \textbf{} • Certification: IT Service Management Foundation based on ISO/IEC 20000 and ISO/IEC 27002\\
% \textbf{} • Certification: OSSTMM Professional Security Tester\\
% \textbf{} • Certification: Red Hat certified system administrator

%----------------------------------------------------------------------------------------
%   WORK EXPERIENCE SECTION
%----------------------------------------------------------------------------------------

\section{Experience}
\begin{entrylist}
%------------------------------------------------
\entry
{Mar. 2016 - present}
{ Yoox-Net-A-porter Group}
{Bologna, Italy}
{\emph{DevOps Engineer}
\begin{itemize}
\item {Public cloud infrastructure build and deploy using several Amazon Web Service stack (EC2, VPC, IAM) and focusing on high-availability, fault tolerance, auto-scaling}
\item {Infrastructure as code and configuration management modules development with respectively Terraform and Ansible. Jenkins pipelines drive all the build and deployment phases of the code}
\item {Implementation of the security layer into the Enterprise Service Bus components to ensure channel authentication, sensitive messages encryption and role-based authorizations on message queue objects}
\item {Instrumentation of the Order Management System and Enterprise Service Bus services with ad hoc sensors for the integration into the local monitoring and alerting system}
\item {Coordinating the local DevOps community to build and share DevOps practices and spread them into our company}
\end{itemize}}
%------------------------------------------------
\end{entrylist}

\begin{entrylist}
%------------------------------------------------
\entry
{2004 - 2016}
{Italian Institute for Nuclear Physics}
{Bologna, Italy}
{\emph{System engineer}
\begin{itemize}
\item {Deliver the deployment, monitoring and log analysis service for product teams}
\item {Build a reliable and fully automated deploy process for any cluster and on-premise environments with Foreman and Puppet}
\item {Operate timely and reliable support: deploy, upgrade, troubleshoot}
\item {Monitoring system with comprehensive checks and alerts}
\item {Making sense of data logs with Elastic stack (Elasticsearch, Logstash and Kibana)}
\item {Assessing security for operated services based on the OSSTMM methodology}
\item {Hardening and securing critical assets in agreement with local policy and NIST/SANS guidelines}
\item {Formalization of a framework of security policies and procedures, based on ISO27002 code of practice}
\item {Scaling and adapting the IT infrastructure to the scientific communities requirements and the evolution of IT technologies and trends}
\end{itemize}}
%------------------------------------------------
\end{entrylist}

%----------------------------------------------------------------------------------------
%   SKILLS SECTION
%----------------------------------------------------------------------------------------
\section{Skills}
\textbf{} • Attention to details in dealing with IT servers availability and reliability\\
\textbf{} • Attitude at documenting and reporting both technical activities and working projects milestones/deliverables as well as using working group methodologies and tracking tools to analyse and prioritize tasks and issues\\
\textbf{} • Experience in implementing helpdesk activity, incident response and problem solving approaches\\
\textbf{} • Large acquaintance in the deployment, monitoring and troubleshooting of Linux systems regarding specifically business continuity, performance optimization and information security\\
\textbf{} • Confidence with DevOps peculiarities for dealing with fast-moving environment\\
\textbf{} • Passionate at scouting innovative technologies and tools\\
\textbf{} • Experience at elaborating and presenting technology research contents and outcomes\\

% \section{Technologies}
% \ztable{System}{Linux, Windows}
% \ztable{Virtualization}{KVM, oVirt}
% \ztable{Cloud}{Amazon AWS, OpenStack}
% \ztable{Lifecycle}{Foreman, Puppet}
% \ztable{Monitoring}{Nagios, Sensu, Zabbix}
% \ztable{Log management}{Elasticsearch, Logstash, Kibana}
% \ztable{Web server and load balancing}{Apache, HAProxy}
% \ztable{RDBMS}{PostgreSQL, MySQL}
% \ztable{Scripting}{Pyhthon, Ruby}
% \ztable{Best practices}{ISO/IEC 20000 and 27002, NIST/SANS}
% \ztable{Testing and version control system}{Vagrant, Git}


%----------------------------------------------------------------------------------------
%   CERTIFICATION SECTION
%----------------------------------------------------------------------------------------
\section{Certification}
\begin{entrylist}
%------------------------------------------------
\entry
{2014}
{IT Service Management Foundation based on ISO/IEC 20000}
{Exin}
{SO/IEC 20000 provides a set of requirements against which an organization can be assessed for effective service management processes}
%------------------------------------------------
\entry
{2011}
{IT Service Management Foundation based on ISO/IEC 27002}
{Exin}
{Implementation of ISO/IEC 27002 best practice and guidelines for vulnerability assessment, risk analysis and deployment of physical, technical and organisational measures for the information security}
%------------------------------------------------
\entry
{2010}
{OSSTMM Professional Security Tester}
{ISECOM}
{The Open Source Security Testing Methodology Manual (OSSTMM) professional security tester is a best practice certification designed to improve methodologies carrying out security tests to assess vulnerability into IT infrastructures}
%------------------------------------------------
\entry
{2009}
{Red Hat certified system administrator}
{Red Hat inc.}
{The Red Hat Certified system administrator aims at certifying the skills on Linux system administration concerning the Red Hat Enterprise Linux (RHEL) operating system}
%------------------------------------------------
\end{entrylist}


%----------------------------------------------------------------------------------------
%   EDUCATION SECTION
%----------------------------------------------------------------------------------------
\section{Education}
\begin{entrylist}
%------------------------------------------------
\entry
{2003}
{Master {\normalfont of Internet Engineering}}
{University of Florence}
{The master covered the main principles on design, project and management of local and wide area networks, Internet based services and web applications}
%------------------------------------------------
\entry
{2002}
{Degree {\normalfont of Astronomy}}
{University of Bologna}
{Knowledge of the main principles of astronomy, mathematics, physics and computing science for the scientific and technological research}
%------------------------------------------------
\end{entrylist}


%----------------------------------------------------------------------------------------
%   INTERESTS SECTION
%----------------------------------------------------------------------------------------

\section{Interests}

\textbf{professional:} IT management, devops weekly news, node lifecycle management, monitoring, information security\\
\textbf{personal:} cooking and making the traditional bolognese pasta down, cycling, contemporary art, joining music and arts festivals


\end{document}
